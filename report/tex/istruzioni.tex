%! Author = Raffaele
%! Date = 17/01/2024

\thispagestyle{headings}
\newpage
\section{Manuale istruzioni sull'esecuzione}\label{sec:istruzioni}
\textit{Assicurarsi di avere Python installato su ogni macchina, il progetto è stato sviluppato con Python 3.11.4} \\ \\
L'applicazione può essere eseguita come segue:
\begin{itemize}
    \item Clonare la repository:
        \begin{verbatim}
            git clone
            https://github.com/rt2013G/reti9-network-pong.git
            && cd reti9-network-pong
        \end{verbatim}
    \item Creare un virtual environment ed attivarlo:
        \begin{verbatim}
            virtualenv -p python3 ./venv
            && source ./venv/bin/activate
        \end{verbatim}
    \item Installare i package necessari:
        \begin{verbatim}
            pip install -r requirements.txt
        \end{verbatim}
    \item Avviare l'applicazione sulla macchina su cui utilizzare la racchetta di \textbf{destra}. \\
    Questo passaggio è importante in quanto di default il seeder iniziale è la racchetta sinistra. \\
        \begin{verbatim}
            python3 pong.py 1 <IP controparte>
        \end{verbatim}
        oppure, se i due processi si trovano sulla stessa macchina:
        \begin{verbatim}
            python3 pong.py 1
        \end{verbatim}
        L'applicazione inserirà automaticamente l'indirizzo di loopback in questo caso.
    \item Avviare l'applicazione sulla macchina su cui utilizzare la racchetta di \textbf{sinistra}. \\
    Attenzione: il gioco inizierà automaticamente
        \begin{verbatim}
            python3 pong.py 0 <IP controparte>
        \end{verbatim}
        oppure, se i due processi si trovano sulla stessa macchina:
        \begin{verbatim}
            python3 pong.py 0
        \end{verbatim}
        L'applicazione inserirà automaticamente l'indirizzo di loopback in questo caso.
\end{itemize}
