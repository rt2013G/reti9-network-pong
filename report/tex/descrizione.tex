%! Author = Raffaele
%! Date = 16/01/2024

\newpage
\thispagestyle{headings}
\section{Descrizione del progetto}\label{sec:Descrizione}
L'obiettivo del progetto è quello di implementare un gioco basato su \textbf{Pong},
tramite comunicazione in tempo reale con UDP. \\
Pong, uno dei primissimi videogiochi arcade, è un simulatore di ping-pong bidimensionale:
i due giocatori controllano una racchetta ciascuno, situate ai due lati dello schermo,
e si scambiano una pallina che si muove da un lato all'altro dello schermo.
Nel momento in cui un giocatore non riesce a intercettare la pallina l'avversario guadagna un punto. \\

Il progetto è stato sviluppato utilizzando il linguaggio di programmazione \textbf{Python}.
La griglia di gioco è stata rappresentata come una griglia di caratteri, che possono essere vuoti,
occupati da una racchetta oppure occupati dalla pallina.
E' stata inoltre utilizzata la libreria \textbf{Pygame} per gestire l'input da tastiera dell'utente
e la rappresentazione a schermo della griglia di gioco.