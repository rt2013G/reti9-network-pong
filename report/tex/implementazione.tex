%! Author = Raffaele
%! Date = 17/01/2024

\thispagestyle{headings}
\newpage
\section{Dettagli implementativi}\label{sec:implementazione}

\subsection{Implementazione dei Peer}\label{subsec:implementazione-dei-peer}

La classe Peer è stata progettata nel seguente modo:
\begin{itemize}
    \item Il costruttore della classe crea la socket UDP ed effettua il bind su tutte le interfacce, utilizzando
    una porta di default arbitaria. \\
    Vengono poi create le variabili degli oggetti utilizzati all'interno del gioco, l'implementazione
    delle classi di tali oggetti è presente nel file
    \textit{app/components.py}
    \item L'indirizzo dell'altro peer viene passato come argomento da linea di comando ed inserito opportunamente.
    \item Il metodo send\_data trasforma l'oggetto Python da inviare in uno stream di byte attraverso il modulo
    \textit{pickle}.
    \item Il metodo \textit{receive\_data} gestisce l'ottenimento dei dati dalla \textit{recvfrom}, trasformando lo
    stream di byte in un oggetto Python. \\
    E' importante inserire un timeout, in quanto è imperativo che la recvfrom non blocchi indefinitivamente il processo:
    il timeout è stato impostato tramite una costante a \textit{5.5 ms}, questo numero deriva dal fatto che se si vuole
    mantenere il gioco a 60 \textit{frame al secondo}, è necessario che ogni frame non richieda più di \textit{16.6 ms}.
    Poiché in ogni frame un Peer al più deve ricevere informazioni sulla pallina, sulla racchetta dell'altro giocatore
    e sullo score, il valore massimo di attesa non può essere più elevato.
    \item Il metodo \textit{receive\_and\_replace\_object\_data} richiama il metodo \textit{receive\_data} ed effettua
    l'azione appropriata a seconda dell'oggetto ritornato.
        \begin{itemize}
            \item Se viene ritornato un oggetto, si aggiorna opportunamente la variabile relativa.
            \item Se la chiamata va in \textit{timeout}, la funzione ritorna. \\
            Il timeout viene effettivamente ignorato in quanto è importante che il gioco prosegui, uno spostamento
            di un frame mancato non è particolarmente impattante sull'applicazione rispetto ad un processo che si blocca
            per assicurarsi che ogni pacchetto inviato sia ricevuto.
            \item Se vi è un altro errore, solitamente causato dalla mancanza dell'altro peer, il processo semplicemente
            va in sleep per un secondo prima di ritentare.
        \end{itemize}
\end{itemize}

\begin{figure}
    \begin{verbatim}
    # import omessi per semplicità

        class Peer:
    def __init__(self, paddle_id):
        # viene creata una socket associata al peer
        # il bind viene effettuato su '', che è l'equivalente di INADDR_ANY
        self.peer_socket = socket.socket(socket.AF_INET, socket.SOCK_DGRAM)

        # come porta viene utilizzata una porta di default arbitaria
        # a cui viene aggiunto l'id del paddle
        self.peer_socket.bind(('', DEFAULT_PORT + paddle_id))

        # l'id è 0 (racchetta sinistra) o 1 (racchetta destra)
        self.id = paddle_id

        # inizializzazione delle variabili utilizzate dalla classe
        self.other_peer = ('', DEFAULT_PORT + paddle_id)
        self.controlled_paddle = Paddle()
        self.other_peer_paddle = Paddle()
        self.ball = Ball()
        self.scorekeeper = Scorekeeper()

    # l'indirizzo dell'altro seed viene passato
    # come argomento da riga di comando
    def set_other_peer(self, address):
        if self.id == LEFT_PADDLE_ID:
            self.other_peer = (address, DEFAULT_PORT + RIGHT_PADDLE_ID)
        else:
            self.other_peer = (address, DEFAULT_PORT + LEFT_PADDLE_ID)
    \end{verbatim}
    \caption{Implementazione del costruttore della classe Peer}
    \label{fig:peer1}
\end{figure}

\begin{figure}
    \begin{verbatim}
    # invia dati all'altro peer
    # trasforma l'oggetto in uno stream di byte con pickle
    def send_data(self, object_data):
        data = pickle.dumps(object_data)
        self.peer_socket.sendto(data, self.other_peer)

    # riceve dati dall'altro peer dopo aver inserito il timeout
    # ritorna un oggetto Python ottenuto dallo stream di byte con pickle
    def receive_data(self):
        self.peer_socket.settimeout(TIMEOUT)
        data, address = self.peer_socket.recvfrom(MAX_BUF)
        object_data = pickle.loads(data)
        return object_data

    # ottiene un oggetto chiamando il metodo receive_data
    # ed effettua l'operazione appropriata
    def receive_and_replace_object_data(self):
        object_data = object
        try:
            object_data = self.receive_data()

        # se la recvfrom va in timeout la funzione semplicemente ritorna
        except socket.timeout:
            return

        # se l'altro processo non è presente, aspetta
        except socket.error:
            print('waiting for the other peer...')
            time.sleep(1)

        # se riceve un oggetto correttamente
        # effettua l'operazione appropriata
        if type(object_data) is Ball:
            self.ball = object_data
        elif type(object_data) is Paddle:
            self.other_peer_paddle = object_data
        elif type(object_data) is Scorekeeper:
            self.scorekeeper = object_data
    \end{verbatim}
    \label{fig:peer2}
\end{figure}

